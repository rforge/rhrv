% !Rnw root = ../../tutorial.Rnw
\chapter{A 15-minutes guide to RHRV\label{ch:Quick}}
% !Rnw root = 15guide.Rnw
In this chapter, a brief description of the RHRV package is presented \cite{vilaRHRV}. Due to the large collection of features that RHRV offers, in this chapter we shall refer only to the most important functionality for performing  a basic \gls{HRV} analysis. In the next chapter we will present more advanced functionality of the package, or functionality geared to certain particular types of analysis. RHRV can be freely downloaded from the R-CRAN repository \cite{cran}.\\

We propose the following basic program flow to perform \gls{HRV} spectral analysis using the RHRV package:

\begin{enumerate}
\item Load heart beat positions. For the sake of simplicity, in this section we will focus in
ASCII files. 
\item Build the instantaneous \gls{HR} series and filter it to eliminate spurious points.
\item Plot the instantaneous \gls{HR} series.
\item Interpolate the instantaneous \gls{HR} series to obtain a \gls{HR} series with equally spaced values.
\item Plot the interpolated \gls{HR} series.
\item Perform the desired analysis. The user can perform time-domain analysis, frequency-domain analysis or nonlinear analysis.
\item Plot the results of the analysis that has been performed and access the ``raw" data.
\end{enumerate}
In section \ref{sec:quickPrep} we will address points 1-5, whereas that in section \ref{sec:quickAna} we will deal with points 6 and 7. All the examples of this chapter will use the 
example beats file ``example.beats" that may be downloaded from \href{http://rhrv.r-forge.r-project.org/}{http://rhrv.r-forge.r-project.org/}. Optionally, the data from this file has been included in RHRV. The user can access this data executing:

\begin{Schunk}
\begin{Sinput}
> # HRVData structure containing the heart beats 
> data("HRVData")
> # HRVData structure storing the results of processing the 
> # heart beats: the beats have been filtered, interpolated, ... 
> data("HRVProcessedData")
\end{Sinput}
\end{Schunk}
The example file is an ASCII file that contains the beats positions obtained from a 2 hours \gls{ECG} (one beat position per row). The subject of the \gls{ECG} is a patient suffering from paraplegia and hypertension (systolic blood pressure above 200 mmHg).  During the recording, he is supplied with prostaglandin E1 (a vasodilator that is rarely employed) and systolic blood pressure fell to 100 mmHg for over an hour. Then, the blood pressure increased slowly up to approximately
 150 mmHg. 

The console output shall be shown for every example.
% !Rnw root = 15guide.Rnw
\section{Preprocessing the Heart Rate series\label{sec:quickPrep}}
\subsubsection{Load heart beat positions} RHRV uses a custom data structure called \textit{HRVData} to store all HRV information related to the signal being analyzed. \textit{HRVData} is implemented as a list object in R language. This list contains all the information corresponding
to the imported signal to be analyzed, some parameters generated
by the pre-processing functions and the HRV analysis results.  A new \textit{HRVData} structure is created using the \textit{CreateHRVData} function. In order to obtain detailed information about
the operations performed by the program, we can activate a verbose mode using the \textit{SetVerbose} function.
\begin{Schunk}
\begin{Sinput}
> hrv.data  = CreateHRVData()
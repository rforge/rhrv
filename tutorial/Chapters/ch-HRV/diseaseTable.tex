\begin{center}
\scriptsize
\begin{longtable}{|p{4cm}|p{4cm}|p{4cm}|}
\caption[Clinical value of HRV analysis in cardiological diseases]
{Summary of the clinical value of HRV analysis in cardiological diseases. Inspired by \cite{forceHRV}.} \label{t:diseases} \\

\hline \multicolumn{1}{|c|}{\textbf{Disease state}} & \multicolumn{1}{c|}{\textbf{Clinical finding }} & \multicolumn{1}{c|}{\textbf{Potential value}} \\ \hline 
\endfirsthead

\multicolumn{3}{c}%
{{\bfseries \tablename\ \thetable{} -- continued from previous page}} \\
\hline \multicolumn{1}{|c|}{\textbf{Disease state}} &
\multicolumn{1}{c|}{\textbf{Clinical findings}} &
\multicolumn{1}{c|}{\textbf{Potential value}} \\ \hline 
\endhead

\hline \multicolumn{3}{|c|}{{Continued on next page}} \\ \hline
\endfoot

\hline \hline
\endlastfoot
Myocardial infarction (MI)&$\downarrow$ HRV after myocardial infarction (MI). In the severe phase of MI, there is a $\downarrow$ standard deviation of the HRV signal& Depressed HRV is a powerful predictor of mortality and of arrhythmic
complications in patients following acute MI\\ \cline{3-3}
& &HRV analysis is useful for risk stratification of patients following MI\\ \hline

Diabetic neuropathy& $\downarrow$ time-domain parameters of HRV 
preceded the clinical detection of autonomic neuropathy. $\downarrow$ LF and HF bands  in diabetic patients with no signs of 
autonomic neuropathy& HRV analysis may be used as predictor of diabetic autonomic neuropathy occurrence\\ \hline
Hypertension&$\uparrow$ LF found in hypertensives with circadian patterns& Hypertension is characterized by depressed circadian rhythmicity of LF\\ \cline{2-3}
            &Reduced parasympathetic activity in hypertensive patients&\\ \hline
Congestive heart failure (CHF)&$\downarrow$ spectral power in all frequencies, especially $>0.04$ Hz&In CHF, there is $\downarrow$ vagal, but relatively
preserved sympathetic modulation of HR\\ \cline{2-3}
		&Low HRV&Reduced vagal activity in CHF patients\\ \cline{2-3}
		&$\downarrow$ HF power in CHF.$\uparrow$ LF/HF& Low parasympathetic tone in CHF. CHF produces imbalance of autonomic tone with $\downarrow$ parasympathetic and predominance of sympathetic tone\\\cline{2-3}
		&Alterations in HRV not tightly linked to severity of CHF. $\downarrow$ HRV was related to sympathetic excitation& \\ \cline{2-3}
		&$\uparrow$ HRV during ACE (angiotensin-converting-enzyme) inhibitor treatment&Increase of the sympathetic tone associated with ACE inhibitor therapy\\ \hline
Heart Transplantation&HRV from $0.02$ to $1$ Hz is $90\%$ reduced& Patients with rejection show less variability\\ \hline
Chronic mitral regurgitation&HR techniques correlated with ventricular performance and predicted clinical events&Prognostic indicator of atrial fibrillation, mortality and progression to valve surgery\\ \hline
Mitral Valve prolapse (MVP)&$\downarrow$ HF power&MVP patients had low vagal tone\\ \hline
Cardiomyopathies&Global and specific vagal tone measurements of HRV were $\downarrow$ in symptomatic patients&\\ \hline
Sudden death (SD) or cardiac arrest (CA)&LF power and standard deviation of HRV signals were related to 1 year mortality&HRV is useful to risk stratify CA survivors for 1 year mortality\\ \cline{2-3}
	&$\downarrow$ HF power in CA survivors&\\ \cline{2-3}
	&Both time and frequency domain indexes separated controls from SD patients. $\downarrow$ HF power was the best separator between heart disease patients
	with and without SD& HF power may be useful predictor of SD\\\cline{2-3}
	&SDNN index was lower in SD patients&Time domain indexes may identify increased risk of SD\\ \hline
Ventricular arrhythmias&HRV indexes do not change consistently before ventricular fibrillation (VF). All power spectra of HRV were significantly $\downarrow$ before the onset of sustained ventricular tachycardia (VT) than before non sustained VT & A temporal relation exists between the decrease of HRV and the onset of sustained VT \\ \hline
\end{longtable}
\end{center}


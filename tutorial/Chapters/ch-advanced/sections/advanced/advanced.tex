% !Rnw root = ../../advanced.Rnw
\chapter{Advanced use of  RHRV\label{ch:RHRV}}

\section{Reading several file formats\label{sec:fileFormats}}
% !Rnw root = advanced.Rnw
 RHRV provides a lot of functionality for importing data files containing heart 
beats positions. Supported formats include ASCII (\textit{LoadBeatAscii} 
function), EDF (\textit{LoadBeatEDFPlus}), Polar (\textit{LoadBeatPolar}), 
Suunto (\textit{LoadBeatSuunto}) and WFDB data files (\textit{LoadBeatWFDB}) 
\cite{mitbih}. We have already dealt with the \textit{LoadBeatASCII} function. 
In this section, we will discuss the remaining functions for reading heart beat 
data.

\subsection{Reading \textit{RR} files} There exist another \textit{RHRV} 
function that reads ASCII files:  the \textit{LoadBeatRR} function. This 
function reads ASCII files storing the \textit{RR} intervals (and not the heart 
beat times). The parameters of the \textit{LoadBeatRR} function are exactly the 
same as those of the \textit{LoadBeatAscii} function, that is:
\begin{Schunk}
\begin{Sinput}
> LoadBeatRR(HRVData, RecordName, RecordPath=".", scale = 1, 
+       datetime = "1/1/1900 0:0:0")
\end{Sinput}
\end{Schunk}
\subsection{Reading files in WFDB format} PhysioNet \cite{physioNet} is a free 
web resource that provides large collections of recorded physiologic signals 
(PhysioBank) and related open-source software (PhysioToolkit). In most cases, a 
record from PhysioBank consists of at least three files, which are named using 
the record name followed by different extensions that indicate their content. 
Almost all records include a binary .dat  file, containing digitized samples of 
one or more signals. The .hea (header) file is a short text file that describes 
the signals. Most records include one or more binary annotation files. For 
example, .qrs files contain an annotation for each QRS complex (heart beat) in 
the recording; .apn files contain apnea annotations; etc. For the sake of 
simplicity, we call ``WFDB file" (WaveForm DataBase) to such a collection of 
files containing data on the same recording \cite{mitbih}. Further details 
about PhysioBank can be found in the Physionet website. \\

The RHRV package provides the \textit{LoadBeatWFDB} function for reading WFDB 
files. This function takes as input parameters the name of the WFDB file to be 
used without any extension (\textit{RecordName} argument), the  relative path 
of the file (\textit{RecordPath} argument) and the extension of the file with 
the heart beats annotations (\textit{annotator}, its default value is ``qrs", 
so in most cases the user won't have to specify it). As an example, we are 
going to
 create a data structure that will read the ``a03" register from the 
PhysioBank's Apnea-ECG database \cite{penzel2000apnea}.
\begin{Schunk}
\begin{Sinput}
> hrv.wfdb = CreateHRVData()
% Rnw root = advanced.Rnw
Intervals of the \gls{HR} time series with pathophysiological interest may be 
annotated in the so-called episode files. For example, it may be interesting to 
compare the heart rate series before, during and after an apnea episode (apneas 
are cessations of a patient's respiratory airflow during the nocturnal rest) 
\cite{lado2012nocturnal}. Such a study could be useful for searching for 
significant differences
in the HRV caused by the apneas. The RHRV package provides functions for 
loading episode information. The supported formats for this information are 
either ASCII (\textit{LoadEpisodesAscii}) or
WFDB (\textit{LoadApneaWFDB}). Episodes may also be added programmatically to 
the time series using the \textit{AddEpisodes} function. All episodes are 
stored under the \textit{Episodes} field of the \textit{HRVData} structure (see 
Figure \ref{fig:completeData}). The plotting functions allow the user to 
include episodic information in the graphics. We will discuss all this points 
in more detail below.\\
\subsection{AddEpisodes\label{par:AddEpisodes}} The simplest way of adding 
episodic information is using the \textit{AddEpisodes} function. 
\textit{AddEpisodes} adds information of episodes by specifying the initial 
times of each episode (\textit{InitTimes} argument, in seconds), the names of 
the episodes (\textit{Tags}), the duration of each episode (\textit{Durations}, 
in seconds) and a numerical identifier for each episode(\textit{Values}). The 
\textit{Values} field is useful for those episodes
that store some numerical values. For example, an apnea episode could store 
information
about the Oxygen saturation level in the \textit{Values} field. Note that all 
the parameters specified by the user will be stored in the \textit{HRVData} 
structure in its corresponding fields as shown in Figure 
\ref{fig:completeData}.\\

Let us read our example file ``example.beats" and add three episodes to it: a 
first type ``A" episode in the $[700,1600]\;s$ interval; a second episode of 
the same type as the first one (``A") in the $[5000,5600]\;s$ interval; and a 
third episode in the $[2000,4500]\;s$ interval of type ``B":

\begin{Schunk}
\begin{Sinput}
> hrv.data = CreateHRVData( )